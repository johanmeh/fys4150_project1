\documentclass{article}
\usepackage[utf8]{inputenc}


\usepackage{enumerate}
\usepackage{amsmath} % Tillater avansert formatering av matte.
\usepackage{amsfonts} % Tillater avanserte teikn, som R for reelle tall.
\usepackage{graphicx} % Tillater mer avansert formatering av grafikk.
\usepackage{geometry} % Tillater enklere formatering av sidevisning.
%\usepackage{physics}
%\usepackage{amssymb}
%\setlength\parindent{0pt}

\title{Project 1}
\author{Johanne Mehren, Stine Sagen and Marit Kollstuen}


\begin{document}


\begin{abstract}
We present our Ferrari algorithm for solving linear equations. Our best algorithm runs as $4n$ FLOPS with $n$ the dimensionality of the matrix.
\end{abstract}


\maketitle

\section{Introduction}
The main goal of project 1 is to develop a general algorithm for solving linear equations and implement a specific matrix to our code. \\
By testing two different ways of solving linear equations, we get a better understanding of the concept of CPU time, how to handle FLOPS and the relative error between the methods.  \\
The report gives a description of the theory behind the algorithm, presenting of the results and a conclusion. \\
For our project, we have used Python programming language which does not require dynamic memory allocation. 

\section{Theory, algorithms and methods}
\subsection{Exact solution}
The one-dimensional Poisson equation \begin{equation} -u''(x) = f(x) \end{equation} does have an exact (analytical) solution of the form: \begin{equation} u(x) = 1 - (1-e^{-10})x - e^{-10x} \end{equation} when assuming the source term on the right hand side of the Poisson equation is \begin{equation} f(x) = 100e^{-10x}. \end{equation} Substituting the above solution into our differential equation, we can then verify that this turns out to be correct. 

\begin{align*}
u(x)& = 1 - (1-e^{-10})x - e^{-10x}  \\
u'(x)& = -1 + e^{-10} + 10e^{-10x} \\
u''(x)& = -100e^{-10x}  \\
-(-100e^{-10x})& = 100e^{-10x} \\
\therefore \\
100e^{-10x}& = 100e^{-10x}
\end{align*}

<<<<<<< HEAD
As expected, equation \eqref{eq2} is an exact solution of the Poisson equation \eqref{eq1}. 

\medskip

Our differential equation concernes a boundary value problem which means our solution also needs to satisfy the given boundary conditions given by:
\begin{equation}\label{eq4}
u(0) = u(1) = 0
\end{equation}
=======
As expected, $u(x) = 1 - (1-e^{-10})x - e^{-10x}$ is an exact solution of $-u''(x) = f(x)$ .\\
>>>>>>> 06df68cd437dd9fdb5e81fbdf770cfab01b0a998

Our differential equation conserns a boundary value problem which means our solution also needs to satisfy the given boundary conditions. 

Checking the behavior of the exact solution at its bondary points 0 and 1: 

\begin{align*}
u(0)& = 1-(1-e^{-10})\cdot 0 -e^{-10\cdot 0} = 0 \\
u(1)& = 1-(1-e^{-10})\cdot 1 -e^{-10\cdot 1} = 0 \\
\end{align*}

<<<<<<< HEAD
Equation \eqref{eq2} is a solution of our boundary value problem.

\medskip

As mentioned in the introduction, we want to compare this exact solution with the numerical solution we obtain when the boundary value problem takes a discretized form. 

\subsection{Finite difference method}

We want to solve the Poisson equation \eqref{eq1} numerically. This can be achived by using something we call finite-difference methods. When using finite-difference methods, we are replacing the derivatives appearing in the differential equation by finite difference approximations at a given set of discrete points in space and/or time. When this method is applied to the second order derivative in the Poisson equation \eqref{eq1}, we obtain a dicretized approximation for u in the x-direction because its only depended on the spacial variable x.  

\medskip

A way to find these finite difference approximations is by using Taylor series expansion on the function $u(x)$. 

\subsubsection{Taylor series expansion}

The derivative of a function u(x) is defined as:

\begin{equation}
\frac{dfu(x)}{dx} = \lim_{h \to 0} \frac{u(x+h) - u(x)}{h}
\end{equation}

where h is the step length. Further a Taylor expansion forward and backward in the spacial direction x can be done: 

\begin{align*}
u(x+h)& = u(x) + hu' + \frac{h^2u''}{2!}  + \frac{h^3u'''}{3!} + O(h^4)\\
u(x-h)& = u(x) - hu' +  \frac{h^2u''}{2!} -\frac{h^3u'''}{3!} + O(h^4)
\end{align*}

where the first equation is expanded forward and the second backward in space.  

\medskip

To obtain finite difference approximation for the second order derivative we then sum the two equations:

=======
$u(x) = 1 - (1-e^{-10})x - e^{-10x}$ is indeed a solution of our boundary value problem. 
\\\\
In our project we want to compare this exact solution with the numerical solution we obtain when the boundary value problem takes a discretized form. 

\subsection{Finite difference method}

We want to solve the Poisson equation numerically and this can be achived by something we call finite-difference methods. Finite-difference methods are about replacing the derivatives appearing in the differential equation by finite difference approximations at a given set of discrete points in space and/or time. E.g when this method is applied to the second order derivative in the poisson equation, we obtain a dicretized approximation for u in the x-direction because its only depended on the x variable. In deriving these finite difference approximations, Taylor series expansion might be very useful.

\subsubsection{Taylor series expansion}

We might use Taylor series expansion in order to derive a suitable finite difference approximation for the differential equation in the process for obtaining a numerical solution. 
\\
Assuming our function u(x) is higher order differensible we can preform a Taylor expansion forward and backward in space:
>>>>>>> 06df68cd437dd9fdb5e81fbdf770cfab01b0a998
\begin{align*}
u(x+h)& = u(x) + hu' + \frac{h^2u''}{2!}  + \frac{h^3u'''}{3!} + o(h^4) \\ 
u(x-h)& = u(x) - hu' +  \frac{h^2u''}{2!} -\frac{h^3u'''}{3!} + o(h^4)
\intertext{To obtain finite difference approximation for the second order derivative we then sum the two equations:}
u(x+h) + u(x-h)& = 2u(x) + \frac{2h^2u''}{2!} + o(h^4) \\
\intertext{In the end we solve for  $u'' $ and get:}
u'&' = \frac{u(x+h) + u(x-h) -2u(x)}{h^2} + o(h^2) \\
\end{align*}
<<<<<<< HEAD

where the h inside of the big O-notation is the truncation error.
Solving for $u''$ we then end up with the following equation:

\begin{equation}
u'' = \frac{u(x+h) + u(x-h) -2u(x)}{h^2} + O(h^2)
\end{equation}

We see that the truncation error is of second order $(h^2)$.  This error that arises when truncating the derivatives with Taylor series. Further, this result will now be applied to our spesific boundary value problem. 
=======
>>>>>>> 06df68cd437dd9fdb5e81fbdf770cfab01b0a998

\subsubsection{Discretization of the bondary value problem}

<<<<<<< HEAD
As mentioned above,  the concept of discretizing is about dividing the domain into a finite set of discrete points. We now the discretize $u$ as $v_i$ with grid points $x_i =  ih$  in the range from $x_0= 0$ to $x_{n+1}=1$. The grid spacing is defined as $h = \frac{x_{n+1}-x_0}{n+1} = \frac{1}{n+1}$.  Our approximated finite difference approximation for equation \eqref{eq1} is now:

\begin{equation}\label{eq7}
-\frac{v_{i+1} + v_{i-1} -2v_i}{h^2} = f_i
\end{equation}
=======
As already mentioned the concept of discretizing is about dividing the domain into a finite set of discrete points. We now the discretize $u$ as $v_i$ with grid points $x_i =  ih$  in the range from $x_0= 0$ to $x_{n+1}=1$. The grid spacing is defined as $h = \frac{x_{n+1}-x_0}{n+1} = \frac{1-0}{n+1} = \frac{1}{n+1}$. The boundary conditions are $v_0 = v_{n+1} = 0$. Our approximated finite difference approximation for the poisson equation is now:

\begin{equation} 
-\frac{v_{i+1} + v_{i-1} -2v_i}{h^2} 
\end{equation}



\subsection{A general algorithm for solving the tridiagonal matrix}
\subsection{Specialized algorithm for solving the tridiagonal matrix}
\subsection{Relative error}
\subsection{LU-decomposition}



\section{Project 1 a)}

We are attempting to solve the equation: 
>>>>>>> 06df68cd437dd9fdb5e81fbdf770cfab01b0a998

with the boundary conditions: $v_0 = v_{n+1} = 0$ and $i = 1,...,n$.  

\medskip


By using algebraic methods, the equation \eqref{eq7} must first be rewritten into a set of linear equations. Assuming $n= 4$, it can be shown that this can be represented as a Toepliz-matrix by setting: 

\[
	v
=
\begin{bmatrix}
	 v_1 & v_2 & v_3 & v_4
\end{bmatrix}
\]

In which 

\[
\begin{bmatrix}
	v_1" \\  v_2" \\ v_3" \\ v_4"
\end{bmatrix}
	\approx -\frac{1}{h^2}
\begin{bmatrix}
	-(v_{0} -2v_1 +v_2) \\
	-(v_1 -2v_2 +v_3) \\
	-(v_{2} -2v_3 +v_4) \\
	-(v_{3} -2v_4 +v_5) 
\end{bmatrix}
\]

The boundary conditions are set to $v_0 = v_{n+1} = 0$, so the expression becomes: 

\[
\begin{bmatrix}
	2 & -1 & 0 & 0 \\
	-1 & 2 & -1 & 0 \\
	0 & -1 & 2 & -1 \\
	0 & 0 & -1 & 2
\end{bmatrix}
\begin{bmatrix}
	v_1 \\  v_2 \\ v_3 \\ v_4
\end{bmatrix}
	= -h^2
\begin{bmatrix}
	f_1 \\
	f_2 \\
	f_3 \\
	f_4
\end{bmatrix}
\]

In which $f_i$ is known. 

\section{Project 1 b)}

\section{Project 1 c)}
In this task we implement matrix (LABEL MATRIX) which has identical values on the diagonal and along the non diagonal elements. FIND NUMBER OF FLOPS! 

\section{Project 1 d)}

\section{Project 1 e)}


\section{Conclusion}

\section{References}


\end{document}