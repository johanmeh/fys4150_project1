\documentclass{article}
\usepackage[utf8]{inputenc}


\usepackage{enumerate}
\usepackage{amsmath} % Tillater avansert formatering av matte.
\usepackage{amsfonts} % Tillater avanserte teikn, som R for reelle tall.
\usepackage{graphicx} % Tillater mer avansert formatering av grafikk.
\usepackage{geometry} % Tillater enklere formatering av sidevisning.
\usepackage{physics}
\setlength\parindent{0pt}

\title{Project 1}
\author{Johanne Mehren, Stine Sagen and Marit Kollstuen}


\begin{document}


\begin{abstract}
We present our Ferrari algorithm for solving linear equations. Our best algorithm runs as $4n$ FLOPS with $n$ the dimensionality of the matrix.
\end{abstract}


\maketitle

\section{Introduction}
The main goal of project 1 is to understand the concept of dynamic memory allocation, a memory handling frequently used in programs such as C++ and Fortran. In C++ you have three ways of managing memory; statically, locally or dynamically. By using dynamic memory allocation, we are reserving space for saving data and are able to manage the lifetime of allocated memory. 

\section{Theory, algorithms and methods}

\section{Project 1 a)}

We are attempting to solve the equation: 

\begin{align*}
	-u''(x) & = f(x), x \in (0, 1), u(0) = u(1) = 0
\end{align*}

which can be approximated as:

\begin{equation}
	v_i" \approx -\frac{v_{i-1}- 2v_i + v_{i+1}}{h^2}
\end{equation}

Assumed $n= 4$, it can be shown that this can be represented as a Toepliz-matrix by setting: 

\[
\begin{bmatrix}
	v
\end{bmatrix}
=
\begin{bmatrix}
	 v_1 & v_2 & v_3 & v_4
\end{bmatrix}
\]

For which 

\[
\begin{bmatrix}
	v_1" \\  v_2" \\ v_3" \\ v_4"
\end{bmatrix}
	\approx -\frac{1}{h^2}
\begin{bmatrix}
	-(v_{0} -2v_1 +v_2) \\
	-(v_1 -2v_2 +v_3) \\
	-(v_{2} -2v_3 +v_4) \\
	-(v_{3} -2v_4 +v_5) 
\end{bmatrix}
\]

The boundary conditions are set to $v_0 = v_{n+1} = 0$, so the expression becomes: 

\[
\begin{bmatrix}
	2 & -1 & 0 & 0 \\
	-1 & 2 & -1 & 0 \\
	0 & -1 & 2 & -1 \\
	0 & 0 & -1 & 2
\end{bmatrix}
\begin{bmatrix}
	v_1 \\  v_2 \\ v_3 \\ v_4
\end{bmatrix}
	= -h^2
\begin{bmatrix}
	f_1 \\
	f_2 \\
	f_3 \\
	f_4
\end{bmatrix}
\]

In which $f_i$ is known. 

\section{Project 1 b)}

\section{Project 1 c)}
In this task we implement matrix 

\section{Project 1 d)}

\section{Project 1 e)}


\section{Conclusion}

\section{References}


\end{document}